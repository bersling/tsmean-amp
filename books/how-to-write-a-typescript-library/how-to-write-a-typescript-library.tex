\documentclass[12pt,a4paper]{article}

% ==== PACKAGES ====
\usepackage{graphicx}
\usepackage[parfill]{parskip}
\usepackage{hyperref}
\usepackage{upquote}
\usepackage[font=small,labelfont=bf]{caption}
\usepackage{listings}
\usepackage{color}

%======= LISTINGS =========
\definecolor{dkgreen}{rgb}{0,0.6,0}
\definecolor{gray}{rgb}{0.5,0.5,0.5}
\definecolor{mauve}{rgb}{0.58,0,0.82}

\lstset{frame=tb,
  language=Java,
  aboveskip=3mm,
  belowskip=3mm,
  showstringspaces=false,
  columns=flexible,
  basicstyle={\small\ttfamily},
  numbers=none,
  numberstyle=\tiny\color{gray},
  keywordstyle=\color{blue},
  commentstyle=\color{dkgreen},
  stringstyle=\color{mauve},
  breaklines=true,
  breakatwhitespace=true,
  tabsize=3
}

%===== MACROS ====
\newcommand{\code}[1]{\fbox{\texttt{#1}}}
\renewcommand{\baselinestretch}{1.3}
\newcommand{\filelabel}[1]{{\footnotesize\color{dkgreen}\texttt{#1}}{\vspace{-0.2cm}}}

%====== SETTINGS =====
\renewcommand{\familydefault}{\sfdefault}


% ===== TITILE PAGE ====
\title{TypeScript - front to back}
\author{Daniel Niederberger}
\begin{document}
\begin{titlepage}
	\centering
	\includegraphics[width=0.5\textwidth]{figures/001-bookshelf.png}\par
	% {\scshape\LARGE A TSMEAN Book \par}
	%\vspace{1cm}
	{\scshape\Large Version 0.1\par}
	\vspace{1.5cm}
	{\huge\bfseries How to Write a\\ TypeScript Library\par}


	%${\Large Daniel Niederberger\par}
 %{Founder of tsmean}
	%\vfill
	%reviewed by\par
	%Dr.~Franz \textsc{Ösinger}

	\vfill

% Bottom of the page
	{\large \today\par}
	\vspace{0.5cm}
\includegraphics[width=2.5cm]{figures/tsmean-logo.pdf}

\end{titlepage}

\tableofcontents
\newpage

\section{Fundamentals}

Writing modular code is a good idea. And nothing is more modular than
writing a library. How can you write a typescript library? Well, that's
exactly what this book is about! This tutorial is from \textbf{late
2017} and written for \textbf{Typescript 2.x}!

\subsection{\texorpdfstring{ Setup
tsconfig.json}{Setup tsconfig.json}}\label{step-1-setup-tsconfig.json}

Create a project folder, in this tutorial we'll call it
\texttt{typescript-library}. Then proceed to create a
\texttt{tsconfig.json} in \texttt{typescript-library}. Your
\texttt{tsconfig.json} file should look somewhat like this:

\filelabel{typescript-library/tsconfig.json}
\begin{lstlisting}
{
  "compilerOptions": {
    "module": "commonjs",
    "target": "es5",
    "declaration": true,
    "outDir": "./dist"
  },
  "include": [
    "src/**/*"
  ]
}
\end{lstlisting}

Pretty much like a setup for a non-library project, but with one
important addition: You need to add the "{declaration: true}" flag. This
will generate the typings required for the consumers to use this
library.

\subsection{\texorpdfstring{ Implement your
library}{Implement your library}}\label{step-2-implement-your-library}

Proceed as you would if you wouldn't write a library. Create a
\texttt{src} folder and in that \texttt{src} folder all the files that
make up your heart of your library.

For this demo, we'll setup a silly \texttt{hello-world.ts} file, that
looks like so:

\filelabel{typescript-library/src/hello-world.ts}
\begin{lstlisting}
export namespace HelloWorld {
  export function sayHello() {
    console.log('hi')
  }
  export function sayGoodbye() {
    console.log('goodbye')
  }
}
\end{lstlisting}

\subsection{\texorpdfstring{ Create an index.ts
file}{Create an index.ts file}}\label{step-3-create-an-index.ts-file}

Add an \texttt{index.ts} file to your \texttt{src}~folder. The purpose
of it is to export all the parts of the library you want to make
available for consumers. In our case it would simply be:

\filelabel{typescript-library/src/index.ts}
\begin{lstlisting}
export {HelloWorld} from './hello-world'
\end{lstlisting}

The consumer would be able to use the library later on like so:

\filelabel{someotherproject/src/somefile.ts}
\begin{lstlisting}
import {HelloWorld} from 'hwrld'
HelloWorld.sayHello();
\end{lstlisting}

You see that we have a new name here,~``hwrld'', we haven't seen
anywhere yet. What is this name? It's the name of the library you're
gonna publish to npm also known as the {package name}!

\subsection{\texorpdfstring{ Configure
package.json}{Configure package.json}}\label{step-4-configure-package.json}

The package name is what the consumer going to use in the~imports later
on, so I prefer to not have it too long and if possible with no hyphens
since it's easiest just to type a string when declaring~the library as
an import (an the consumers editor doesn't have auto-import). If you
have to type
\texttt{import\ *\ as\ typescript-library\ from\ \textquotesingle{}my-crazy-library-name\textquotesingle{}}
each time, it's not so pleasant. For this demo I have have chosen
\texttt{hwrld} since it was still available on npm.

The package name is usually right at the top of the
\texttt{package.json}. The whole
\texttt{package.json} would look like so:

\filelabel{typescript-library/package.json}
\begin{lstlisting}
{
  "name": "hwrld",
  "version": "1.0.0",
  "description": "Can log \"hello world\" and \"goodbye world\" to the console!",
  "main": "dist/index.js",
  "types": "dist/index.d.ts"
}
\end{lstlisting}

If you don't have a package yet you can create one with
\texttt{npm\ init} and it will guide you through the process. Remember,
the package name you choose will be the one people later use in their
imports, so choose wisely!

There's also one all-important flag in this package.json: You have to
declare where to find the type declarations! This is done using
{``types'': ``dist/index.d.ts''} Otherwise the consumer won't find your
module!

\subsection{\texorpdfstring{ Configure
.npmignore}{Configure .npmignore}}\label{step-5-configure-.npmignore}

When you look at your package when loading it from npm, you'll notice
that it contains the dist AND the src folder. However, you don't really
need the source folder, as the compiled files along with the typings
(.d.ts files) live in the dist folder. Thus you can create an .npmignore
file. In our case it's simple:

\filelabel{typescript-library/.npmignore}
\begin{lstlisting}
tsconfig.json
src
\end{lstlisting}

\subsection{\texorpdfstring{ Publish to
npm}{Publish to npm}}\label{step-6-publish-to-npm}

To publish your first version to npm run:

\begin{lstlisting}
tsc
npm publish
\end{lstlisting}

Now you're all set to go! Consume your library anywhere you want by
running:

\begin{lstlisting}
npm install --save hwrld
\end{lstlisting}

and consume it using

\begin{lstlisting}
import {HelloWorld} from 'hwrld'
HelloWorld.sayHello();
\end{lstlisting}

For subsequent releases, use the semvar principle. When you make a patch
/ bugfix to your library, you can run \texttt{npm\ version\ patch}, for
new features run \texttt{npm\ version\ minor} and on breaking changes of
your api run \texttt{npm\ version\ major}.


\section{Create a consumer for your
library}\label{create-a-consumer-for-your-library}

On \href{/}{the main tutorial} we've setup the library
\texttt{\textquotesingle{}hwrld\textquotesingle{}}. But let's say we're
not comfortable publishing it to npm yet and we want to test locally how
it would behave once published. Luckily, setting up a consumer for your
library is really easy.

\subsection{\texorpdfstring{ Create a
folder}{Create a folder}}\label{step-1-create-a-folder}

If your library is contained for example in \texttt{mylib}, then create
\texttt{libconsumer} next to it so your folder structure looks like
this:

\begin{lstlisting}
somewhere-on-your-machine
 -- mylib
 -- libconsumer
\end{lstlisting}

\subsection{\texorpdfstring{ Create a
package.json}{Create a package.json}}\label{step-2-create-a-package.json}

Create a \texttt{package.json} file. It could look somewhat like this
one:

\filelabel{libconsumer/package.json}
\begin{lstlisting}
{
  "name": "libconsumer",
  "version": "1.0.0",
  "description": "",
  "main": "index.js",
  "scripts": {
    "test": "echo \"Error: no test specified\" && exit 1"
  },
  "author": "",
  "license": "ISC"
}
\end{lstlisting}

It actually doesn't matter too much what's in there, but it needs to be
present for the next step to work.

\subsection{\texorpdfstring{ Link your
library}{Link your library}}\label{step-3-link-your-library}

Now comes the magic. Run

\begin{lstlisting}
npm link ../mylib
\end{lstlisting}

% Hidden Mistake 3
This will create a symlink in \texttt{libconsumer}'s
\texttt{node\_modules} to the \texttt{mylib} library. To red more on
\texttt{npm\ link}, \href{https://docs.npmjs.com/cli/link}{see the
docs}.

\subsection{\texorpdfstring{{} Consume your
library}{Consume your library}}\label{step-4-consume-your-library}

Write some file(s) to consume your library. For example

\filelabel{libconsumer/somefile.ts}

\begin{lstlisting}
import {HelloWorld} from 'hwrld'
HelloWorld.sayHello();
\end{lstlisting}

You compile this simple example like this:

\begin{lstlisting}
\filelabel{tsc somefile.ts}
\end{lstlisting}

Finally you can run

\begin{lstlisting}
node somefile.js
\end{lstlisting}

and in this example it should log \texttt{hi} to the console. And that's
it!


\section{Unit testing a TypeScript
library}\label{unit-testing-a-typescript-library}

Unit tests are the cornerstone of a bug free application. As your
codebase grows, you can't always test every feature manually. But you
\emph{can} test every feature using unit tests. And it's not even hard
to set up. Here's how to set up unit tests in Typescript!

\subsection{\texorpdfstring{ Install packages for unit
testing}{Install packages for unit testing}}\label{step-1-install-packages-for-unit-testing}

Unit tests \texttt{assert} some properties of the code. For example a
method \texttt{add(x,y)} should correctly add x and y. With a unit test
you can then test some sample cases like
\texttt{assert(add(3,4)).equals(7)}. To get this \texttt{assert} method
and to execute tests in the first place, we need some testing software.
Here we'll choose Mocha and Chai. To install mocha and chai, you can
run:

\begin{lstlisting}
yarn add mocha @types/mocha chai @types/chai ts-node typescript --dev
\end{lstlisting}

% Hidden Mistake 2
You will need too have \texttt{yarn} installed for this to work, see
\href{https://yarnpkg.com/lang/en/docs/install}{install manual}.

\subsection{\texorpdfstring{ Writing your first unit
test}{Writing your first unit test}}\label{step-2-writing-your-first-unit-test}

Let's say you have the following unit:

\filelabel{typescript-library/src/math.ts}

\begin{lstlisting}
export namespace MyMath {
  export function add(x: number, y: number) {
    return x + y;
  }
}
\end{lstlisting}

Then a first unit test could look like:

\filelabel{typescript-library/src/math.test.ts}

\begin{lstlisting}
import { MyMath } from './math';

import * as mocha from 'mocha';
import * as chai from 'chai';

const expect = chai.expect;
describe('My math library', () => {

it('should be able to add things correctly' , () => {
  expect(MyMath.add(3,4)).to.equal(7);
});

});
\end{lstlisting}

\subsection{\texorpdfstring{ Run the unit
test(s)}{Run the unit test(s)}}\label{step-3-run-the-unit-tests}

You can run the test with the following command:

\begin{lstlisting}
mocha --reporter spec --compilers ts:ts-node/register 'src/**/*.test.ts'
\end{lstlisting}

% Hidden Mistake 1
An basically that's it. You should see an output in the console, that
looks like this:

You can put this long command into \texttt{package.json} into
\texttt{"scripts:\ \{\ "test":\ "..."\}"} and then run the tests with
\texttt{npm\ test}. So the complete package.json at this point would
look like:

\begin{lstlisting}
{
  "devDependencies": {
    "@types/chai": "^4.0.0",
    "@types/mocha": "^2.2.41",
    "chai": "^4.0.2",
    "mocha": "^3.4.2",
    "ts-node": "^3.1.0"
  },
  "scripts": {
    "test": "mocha --reporter spec --compilers ts:ts-node/register 'src/**/*.test.ts'"
  }
}
\end{lstlisting}

If you want to run a single test, run
\texttt{mocha\ -\/-reporter\ spec\ -\/-compilers\ ts:ts-node/register\ -\/-grep\ "TestName"\ \textquotesingle{}src/**/*.test.ts\textquotesingle{}}
where ``TestName'' matches any \texttt{describe} or \texttt{it}. What's
also quite cool, with IntelliJ you can setup your unit tests such that
you can execute single tests right from the code!
\href{https://www.jetbrains.com/help/idea/2017.1/run-debug-configuration-mocha.html}{Here
is the manual.}


\section{Building an Angular
Library?}\label{building-an-angular-library}

\subsection{How to write an Angular
library}\label{how-to-write-an-angular-library}



Since AngularCLI nor any of the official Angular resources offer a
straightforward way, I've composed my own tutorial on how to write an
Angular library. \textbf{Note: This tutorial describes how to publish
typescript sources to npm. The library will only work for Angular
projects written in typescript!} I don't compile to javascript plus d.ts
files, because after hours of trying I was on the brink of throwing my
notebook out the window. It was too annoying with rollup, umd,
MyLibraryModule not found, index.ts names, long waiting times with
``yarning'' and ``ng serving'', inlineing html and css, crazy
tsconfig.json's with deprecated flags, no good local development
environment, working version in local project but breaking when
published and so on. But since a majority of the community uses
typescript anyways and it's not my fault that it's so damn hard to
publish a working library, I'm just going to publish typescript sources,
which works like a charm.

\subsection{\texorpdfstring{ Create a new project with the
AngularCli}{Create a new project with the AngularCli}}\label{step-1-create-a-new-project-with-the-angularcli}

Create a new project. This will be a wrapper and consumer for your
library module. I am going to call my library \texttt{libex} (for
``library-example'', and it was still free on npm) so I call the new
project \texttt{libex-project}

\begin{lstlisting}
ng new libex-project  --prefix libex
\end{lstlisting}

Use your library title instead of libex. Prefix is what you'll write in
front of your components, for example if I have a
\texttt{HelloComponent} it will be used by \texttt{libex-hello} now.

\subsection{\texorpdfstring{ Create a new
module}{Create a new module}}\label{step-2-create-a-new-module}

Your library will reside in it's own module. But first we've got to
create said module.

\begin{lstlisting}
ng g module libex
\end{lstlisting}

Then we \texttt{cd} into that folder.

\begin{lstlisting}
cd src/app/libex/
\end{lstlisting}

\subsection{\texorpdfstring{ Build your library
module}{Build your library module}}\label{step-3-build-your-library-module}

Create components, services etc., e.g.

\begin{lstlisting}
ng g component hello
\end{lstlisting}

When you're done, you'll have to \textbf{export} the desired components:

\begin{lstlisting}
@NgModule({
imports: [
CommonModule
],
declarations: [HelloComponent],
exports: [HelloComponent]
})
\end{lstlisting}

You can use your \texttt{AppModule} to test the library:

\begin{lstlisting}
...
imports: [
  BrowserModule,
  LibexModule
],
\end{lstlisting}

If you need singleton services, you should modify your library module
like so:

\begin{lstlisting}
@NgModule({
  providers: [ /* Don't add the services here */ ]
})
export class LibexModule {
  static forRoot() {
    return {
      ngModule: LibexModule,
      providers: [ SomeService ]
    }
  }
}
\end{lstlisting}

and change the imports in \texttt{AppModule} to:

\begin{lstlisting}
...
imports: [
  BrowserModule,
  LibexModule.forRoot()
],
\end{lstlisting}

\subsection{\texorpdfstring{{}
Publish}{Publish}}\label{step-4-publish}

In your module folder, create a new \texttt{package.json}. You could do
this with the \texttt{npm\ init} command. It should look somewhat like
this:

\begin{lstlisting}
{
  "name": "libex",
  "version": "1.0.16",
  "description": "An Example Library.",
  ...
}
\end{lstlisting}

Basically only the library name matters. \textbf{Also rename
\texttt{libex.module.ts} to \texttt{index.ts} since that's the standard
name for a main file.} Now since we're just publishing the typescript
sources (note: that means your library will only work for consumers that
\textbf{also} use typescript), we're ready to publish. Just run
\texttt{npm\ publish}! You can also add a \texttt{.npmignore} so you
publish only exactly what's needed, it works like .gitignore, just for
npm.

What's also an option is to re-export some files from index.ts. For
example
\texttt{export\ *\ from\ \textquotesingle{}./hello.service\textquotesingle{}},
since then the consumer can just write
\texttt{import\ \{\ HelloService\ \}\ from\ \textquotesingle{}libex\textquotesingle{}}
instead of
\texttt{import\ \{\ HelloService\ \}\ from\ \textquotesingle{}libex/hello.service\textquotesingle{}}.
On the other hand, this can also
\href{https://stackoverflow.com/a/38000323/3022127}{lead to problems},
so I usually refrain from it.

In case your library requires other libraries, you can also install them
in your library module's folder and to the \texttt{package.json} we've
just created. Just run \texttt{npm\ install\ -\/-save\ something} from
the library folder.

For subsequent releases, \texttt{npm\ version\ patch} (or \texttt{minor}
or \texttt{major}) and then \texttt{npm\ publish}.

Well and that's it, now you have your library, which you can develop \&
test locally and publish to npm!


\section{Writing a command-line executable with
TypeScript}\label{writing-a-command-line-executable-with-typescript}

On \href{/}{the main tutorial} we've setup the library
\texttt{\textquotesingle{}hwrld\textquotesingle{}}. Now we're going to
learn how to make a library available as a system command. Actually,
once you've completed the main tutorial, this becomes really easy. Just
two simple steps!

\subsection{\texorpdfstring{ Add execution
instructions}{Add execution instructions}}\label{-add-execution-instructions}

On top of your executable files (the main files), add the following
line:

\begin{lstlisting}
#!/usr/bin/env node
\end{lstlisting}

This makes sure the system understands how to execute the compiled
javascript file, by instructing it to interpret it with \texttt{node}.

\subsection{\texorpdfstring{ Modify the
package.json}{Modify the package.json}}\label{step-2-modify-the-package.json}

The \texttt{package.json} just needs to be modified a bit, so it looks
like this:

\filelabel{typescript-library/package.json}

\begin{lstlisting}
{
  "name": "hwrld",
  "version": "1.0.0",
  "description": "Can log \"hello world\" and \"goodbye world\" to the console!",
  "main": "dist/index.js",
  "bin": {
    "hwrld": "dist/index.js"
  },
  "types": "dist/index.d.ts"
}
\end{lstlisting}

And voilà! Once you publish this, you will be able to install you're
package globally on a machine using:

\begin{lstlisting}
sudo npm install -g hwrld
\end{lstlisting}

Of course for the \texttt{hwrld} package this is pretty useless. A
really simple yet instructive example I've written is a controller for
the \texttt{mpc} music player. You can find the code here:
\url{https://github.com/bersling/mpc-control}.

By the way, there wasn't really anything typescripty about this
tutorial, it holds just as well for plain javascript node modules!


\end{document}
